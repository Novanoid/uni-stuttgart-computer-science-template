% !TeX spellcheck = de_DE
% Dieses Dokument muss mit PDFLatex gesetzt werden
% Vorteil: Grafiken koennen als jpg, png, ... verwendet werden
%          und die Links im Dokument sind auch gleich richtig
%
%Ermöglicht \\ bei der Titelseite (z.B. bei supervisor)
%Siehe https://github.com/latextemplates/uni-stuttgart-cs-cover/issues/4
\RequirePackage{kvoptions-patch}

%English:
%\let\ifdeutsch\iffalse
%\let\ifenglisch\iftrue

%German:
\let\ifdeutsch\iftrue
\let\ifenglisch\iffalse

%
\ifenglisch
	\PassOptionsToClass{numbers=noenddot}{scrbook}
\else
	%()Aus scrguide.pdf - der Dokumentation von KOMA-Script)
	%Nach DUDEN steht in Gliederungen, in denen ausschließlich arabische Ziffern für die Nummerierung
	%verwendet werden, am Ende der Gliederungsnummern kein abschließender Punkt
	%(siehe [DUD96, R3]). Wird hingegen innerhalb der Gliederung auch mit römischen Zahlen
	%oder Groß- oder Kleinbuchstaben gearbeitet, so steht am Ende aller Gliederungsnummern ein
	%abschließender Punkt (siehe [DUD96, R4])
	\PassOptionsToClass{numbers=autoendperiod}{scrbook}
\fi

%Warns about outdated packages and missing caption delcarations
%See https://www.ctan.org/pkg/nag
\RequirePackage[l2tabu, orthodox]{nag}

%Neue deutsche Trennmuster
%Siehe http://www.ctan.org/pkg/dehyph-exptl und http://projekte.dante.de/Trennmuster/WebHome
%Nur für pdflatex, nicht für lualatex
\RequirePackage{ifluatex}
\ifluatex
%do not load anything
\else
	\ifdeutsch
		\RequirePackage[ngerman=ngerman-x-latest]{hyphsubst}
	\fi
\fi

\documentclass[
               fontsize=12pt, %Default: 11pt, bei Linux Libertine zu klein zum Lesen
% BEGINN: Optionen für typearea
               paper=a4,
               twoside,  % fuer die Betrachtung am Schirm ungeschickt
               BCOR=3mm, % Bindekorrektur
               DIV=13,   % je höher der DIV-Wert, desto mehr geht auf eine Seite. Gute werde sind zwischen DIV=12 und DIV=15
               headinclude=true,
               footinclude=false,
% ENDE: Optionen für typearea
%               titlepage,
               bibliography=totoc,
%               idxtotoc,   %Index ins Inhaltsverzeichnis
%                liststotoc, %List of X ins Inhaltsverzeichnis, mit liststotocnumbered werden die Abbildungsverzeichnisse nummeriert
               headsepline,
               cleardoublepage=empty,
               parskip=half,
%               draft    % um zu sehen, wo noch nachgebessert werden muss - wichtig, da Bindungskorrektur mit drin
               final   % ACHTUNG! - in pagestyle.tex noch Seitenstil anpassen
               ]{scrbook}


%%%
% Beschreibung:
% In dieser Datei werden zuerst die benoetigten Pakete eingebunden und
% danach diverse Optionen gesetzt. Achtung Reihenfolge ist entscheidend!
%
%%%


%%%
% Styleguide:
%
% Ein sehr kleiner Styleguide. Packages werden in Blöcken organisiert.
% Ein Block beginnt mit drei % in einer Zeile, dann % <Blocküberschrift>, dann
% eine Liste der möglichen Optionen und deren Einstellungen, Gründe und Kommentare
% eine % Zeile in der sonst nichts steht und dann wieder %%% in einer Zeile.
%
% Zwischen zwei Blöcken sind 2 Leerzeilen!
% Zu jedem Paket werden soviele Optionen wie möglich/nötig angegeben
%
%%%

%%%
% Required for recent version of komascript, as this template does not use the most recent commands of KOMAScript
\usepackage{scrhack}
%%%

%%%
% Codierung
% Wir sind im 21 Jahrhundert, utf-8 löst so viele Probleme.
%
% Mit UTF-8 funktionieren folgende Pakete nicht mehr. Bitte beachten!
%   * fancyvrb mit §
%   * easylist -> http://www.ctan.org/tex-archive/macros/latex/contrib/easylist/
\ifluatex
%no package loading required
\else
\usepackage[utf8]{inputenc}
\fi
%
%%%

%%%
%Parallelbetrieb tex4ht und pdflatex
\makeatletter
\@ifpackageloaded{tex4ht}{\def\iftex4ht{\iftrue}}
                         {\def\iftex4ht{\iffalse}}
\makeatother
%%%


%%%
%Farbdefinitionen
\usepackage[hyperref,dvipsnames]{xcolor}
%

%%%
% Required for custom acronyms/glossaries style
% Left aligned Columns in tables with fixed width
% see http://tex.stackexchange.com/questions/91566/syntax-similar-to-centering-for-right-and-left
\usepackage{ragged2e}
%%%

%%%
% Abkürzungsverzeichnis
\usepackage{scrwfile} % Wichtig, ansonsten erscheint "No room for a new \write"
% siehe http://www.dickimaw-books.com/cgi-bin/faq.cgi?action=view&categorylabel=glossaries#glsnewwriteexceeded
\usepackage[acronym,indexonlyfirst,nomain]{glossaries}
\ifdeutsch
\renewcommand*{\acronymname}{Abkürzungsverzeichnis}
\else
\renewcommand*{\acronymname}{List of Abbreviations}
\fi
\renewcommand*{\glsgroupskip}{}
%
% Removed Glossarie as a table as a quick fix to get the template working again
% see http://tex.stackexchange.com/questions/145579/how-to-print-acronyms-of-glossaries-into-a-table
%
\makenoidxglossaries
%%%


%%%
% Neue deutsche Rechtschreibung und Literatur statt "Literature", Nachfolger von ngerman.sty
\ifdeutsch
% letzte Sprache ist default, Einbindung von "american" ermöglicht \begin{otherlanguage}{amercian}...\end{otherlanguage} oder \foreignlanguage{american}{Text in American}
% see also http://tex.stackexchange.com/a/50638/9075
\usepackage[american,ngerman]{babel}
% Ein "abstract" ist eine "Kurzfassung", keine "Zusammenfassung"
\addto\captionsngerman{%
	\renewcommand\abstractname{Kurzfassung}%
}
\else
%
%
% if you are writing in english
% last language is the default language
\usepackage[ngerman,american]{babel}
\fi
%
%%%

%%%
% Anführungszeichen
% Zitate in \enquote{...} setzen, dann werden automatisch die richtigen Anführungszeichen verwendet.
\usepackage{csquotes}
%%%


%%%
% erweitertes Enumerate
\usepackage{paralist}
%
%%%


%%%
% fancyheadings (nicht nur) fuer koma
\usepackage[automark]{scrlayer-scrpage}
%
%%%


%%%
%Mathematik
%
\usepackage[]{amsmath} % Viele Mathematik-Sachen: Doku: /usr/share/doc/texmf/latex/amsmath/amsldoc.dvi.gz
\PassOptionsToPackage{fleqn,leqno}{amsmath} % options must be passed this way, otherwise it does not work with glossaries
%fleqn (=Gleichungen linksbündig platzieren) funktioniert nicht direkt. Es muss noch ein Patch gemacht werden:
%\addtolength\mathindent{1em}%work-around ams-math problem with align and 9 -> 10. Does not work with glossaries, No visual changes.
\usepackage{mathtools} %fixes bugs in AMS math
%
%for theorems, replacement for amsthm
\usepackage[amsmath,hyperref]{ntheorem}
\theorempreskipamount 2ex plus1ex minus0.5ex
\theorempostskipamount 2ex plus1ex minus0.5ex
\theoremstyle{break}
\newtheorem{definition}{Definition}[section]
%
%%%


%%%
% Intelligentes Leerzeichen um hinter Abkürzungen die richtigen Abstände zu erhalten, auch leere.
% siehe commands.tex \gq{}
\usepackage{xspace}
%Macht \xspace und \enquote kompatibel
\makeatletter
\xspaceaddexceptions{\grqq \grq \csq@qclose@i \} }
\makeatother
%
%%%


%%%
% Anhang
\usepackage{appendix}
%[toc,page,title,header]
%
%%%


%%%
% Grafikeinbindungen
\usepackage{graphicx}%Parameter "pdftex" unnoetig
\graphicspath{{\getgraphicspath}}
\newcommand{\getgraphicspath}{graphics/}
%
%%%


%%%
% Enables inclusion of SVG graphics - 1:1 approach
% This is NOT the approach of http://www.ctan.org/tex-archive/info/svg-inkscape,
% which allows text in SVG to be typeset using LaTeX
% We just include the SVG as is
\usepackage{epstopdf}
\epstopdfDeclareGraphicsRule{.svg}{pdf}{.pdf}{%
  inkscape -z -D --file=#1 --export-pdf=\OutputFile
}
%
%%%


%%%
% Enables inclusion of SVG graphics - text-rendered-with-LaTeX-approach
% This is the approach of http://www.ctan.org/tex-archive/info/svg-inkscape,
\newcommand{\executeiffilenewer}[3]{%
\IfFileExists{#2}
{
%\message{file #2 exists}
\ifnum\pdfstrcmp{\pdffilemoddate{#1}}%
{\pdffilemoddate{#2}}>0%
{\immediate\write18{#3}}
\else
{%\message{file up to date #2}
}
\fi%
}{
%\message{file #2 doesn't exist}
%\message{argument: #3}
%\immediate\write18{echo "test" > xoutput.txt}
\immediate\write18{#3}
}
}
\newcommand{\includesvg}[1]{%
\executeiffilenewer{#1.svg}{#1.pdf}%
{
inkscape -z -D --file=\getgraphicspath#1.svg %
--export-pdf=\getgraphicspath#1.pdf --export-latex}%
\input{\getgraphicspath#1.pdf_tex}%
}


%%%
\usepackage{siunitx}
%%%

%%%
% Tabellenerweiterungen
\usepackage{array} %increases tex's buffer size and enables ``>'' in tablespecs
\usepackage{longtable}
\usepackage{dcolumn} %Aligning numbers by decimal points in table columns
\ifdeutsch
	\newcolumntype{d}[1]{D{.}{,}{#1}}
\else
	\newcolumntype{d}[1]{D{.}{.}{#1}}
\fi

%
%%%

%%%
% Eine Zelle, die sich über mehrere Zeilen erstreckt.
% Siehe Beispieltabelle in Kapitel 2
\usepackage{multirow}
%
%%%

%%%
%Fuer Tabellen mit Variablen Spaltenbreiten
%\usepackage{tabularx}
%\usepackage{tabulary}
%
%%%


%%%
% Links verhalten sich so, wie sie sollen
\usepackage{url}
%
%Use text font as url font, not the monospaced one
%see comments at http://tex.stackexchange.com/q/98463/9075
\urlstyle{same}
%
%Hint by http://tex.stackexchange.com/a/10419/9075
\makeatletter
\g@addto@macro{\UrlBreaks}{\UrlOrds}
\makeatother
%
%%%


%%%
% Index über Begriffe, Abkürzungen
%\usepackage{makeidx} makeidx ist out -> http://xindy.sf.net verwenden
%
%%%

%%%
%lustiger Hack fuer das Abkuerzungsverzeichnis
%nach latex durchlauf folgendes ausfuehren
%makeindex ausarbeitung.nlo -s nomencl.ist -o ausarbeitung.nls
%danach nochmal latex
%\usepackage{nomencl}
%    \let\abk\nomenclature %Deutsche Ueberschrift setzen
%          \renewcommand{\nomname}{List of Abbreviations}
%        %Punkte zw. Abkuerzung und Erklaerung
%          \setlength{\nomlabelwidth}{.2\hsize}
%          \renewcommand{\nomlabel}[1]{#1 \dotfill}
%        %Zeilenabstaende verkleinern
%          \setlength{\nomitemsep}{-\parsep}
%    \makenomenclature
%
%%%

%%%
% Logik für Tex
\usepackage{ifthen} %fuer if-then-else @ commands.tex
%
%%%


%%%
%
\usepackage{listings}
%
%%%


%%%
%Alternative zu Listings ist fancyvrb. Kann auch beides gleichzeitig benutzt werden.
\usepackage{fancyvrb}
%\fvset{fontsize=\small} %Groesse fuer den Fliesstext. Falls deaktiviert: \normalsize
%Funktioniert mit UTF-8 nicht mehr
%\DefineShortVerb{\§} %Somit kann im Text ganz einfach |verbatim| text gesetzt werden.
\RecustomVerbatimEnvironment{Verbatim}{Verbatim}{fontsize=\footnotesize}
\RecustomVerbatimCommand{\VerbatimInput}{VerbatimInput}{fontsize=\footnotesize}
%
%%%


%%%
% Bildunterschriften bei floats genauso formatieren wie bei Listings
% Anpassung wird unten bei den newfloat-Deklarationen vorgenommen
% https://www.ctan.org/pkg/caption2 is superseeded by this package.
\usepackage{caption}
%
%%%


%%%
% Ermoeglicht es, Abbildungen um 90 Grad zu drehen
% Alternatives Paket: rotating Allerdings wird hier nur das Bild gedreht, während bei lscape auch die PDF-Seite gedreht wird.
%Das Paket lscape dreht die Seite auch nicht
\usepackage{pdflscape}
%
%%%


%%%
% Fuer listings
% Wird für fancyvrb und für lstlistings verwendet
\usepackage{float}

%\usepackage{floatrow}
%% zustäzlich für den Paramter [H] = Floats WIRKLICH da wo sie deklariert wurden paltzieren - ganz ohne Kompromisse
% floatrow ist der Nachfolger von float
% Allerdings macht floatrow in manchen Konstellationen Probleme. Deshalb ist das Paket deaktiviert.
%
%%%



%%%
% Fuer Abbildungen innerhalb von Abbildungen
% Ersetzt das Paket subfigure
%
% Due to bug #24 in the caption package we need to update caption3.sty at the moment manualy to use subfig.
% Bug #24: http://sourceforge.net/p/latex-caption/tickets/24/
% corrected caption3.sty: http://sourceforge.net/p/latex-caption/code/HEAD/tree/branches/3.3/tex/caption3.sty
%
\usepackage[caption=false, lofdepth=1, lotdepth, margin=5pt]{subfig}
%
%%%




%%%
% Fußnoten
%
%\usepackage{dblfnote}  %Zweispaltige Fußnoten
%
% Keine hochgestellten Ziffern in der Fußnote (KOMA-Script-spezifisch):
%\deffootnote[1.5em]{0pt}{1em}{\makebox[1.5em][l]{\bfseries\thefootnotemark}}
%
% Abstand zwischen Fußnoten vergrößern:
%\setlength{\footnotesep}{.85\baselineskip}
%
%
%
%Folgendes Kommando deaktiviert die Trennlinie zur Fußnote
%\renewcommand{\footnoterule}{}
%
\addtolength{\skip\footins}{\baselineskip} % Abstand Text <-> Fußnote
%
% Fußnoten immer ganz unten auf einer \raggedbottom-Seite
% fnpos kommt aus dem yafoot package
\usepackage{fnpos}
\makeFNbelow
\makeFNbottom
%
%%%


%%%
%
\raggedbottom     % Variable Seitenhöhen zulassen
%
%%%


%%%
% Falls die Seitenzahl bei einer Referenz auf eine Abbildung nur dann angegeben werden soll,
% falls sich die Abbildung nicht auf der selben Seite befindet...
\iftex4ht
%tex4ht does not work well with vref, therefore we emulate vref behavior
\newcommand{\vref}[1]{\ref{#1}}
\else
\ifdeutsch
\usepackage[ngerman]{varioref}
\else
\usepackage{varioref}
\fi
\fi
%%%

%%%
% Noch schoenere Tabellen als mit booktabs mit http://www.zvisionwelt.de/downloads.html
\usepackage{booktabs}
%
%\usepackage[section]{placeins}
%
%%%


%%%
%Fuer Graphiken. Allerdings funktioniert es nicht zusammen mit pdflatex
%\usepackage{gastex} % \tolarance kann dann nicht mehr umdefiniert werden
%
%%%


%%%
%
%\usepackage{multicol}
%\usepackage{setspace} % kollidiert mit diplomarbeit.sty
%
%http://www.tex.ac.uk/cgi-bin/texfaq2html?label=floats
%\usepackage{flafter} %floats IMMER nach ihrer Deklaration platzieren
%
%%%


%%%
%schoene TODOs
\usepackage{todonotes}
\let\xtodo\todo
\renewcommand{\todo}[1]{\xtodo[inline,color=black!5]{#1}}
\newcommand{\utodo}[1]{\xtodo[inline,color=green!5]{#1}}
\newcommand{\itodo}[1]{\xtodo[inline]{#1}}
%
%%%


%%%
%biblatex statt bibtex
\usepackage[
  backend       = biber, %biber does not work with 64x versions alternative: bibtex8
						 %minalphanames only works with biber backend
  sortcites     = true,
  bibstyle      = alphabetic,
  citestyle     = alphabetic,
  firstinits    = true,
  useprefix     = false, %"von, van, etc." will be printed, too. See below.
  minnames      = 1,
  minalphanames = 3,
  maxalphanames = 4,
  maxbibnames   = 99,
  maxcitenames  = 3,
	natbib        = true,
	eprint        = true,
	url           = true,
  doi           = true,
  isbn          = true,
  backref       = true]{biblatex}
\bibliography{bibliography}
%\addbibresource[datatype=bibtex]{bibliography.bib}

%Do not put "vd" in the label, but put it at "\citeauthor"
%Source: http://tex.stackexchange.com/a/30277/9075
\makeatletter
\AtBeginDocument{\toggletrue{blx@useprefix}}
\AtBeginBibliography{\togglefalse{blx@useprefix}}
\makeatother

%Thin spaces between initials
%http://tex.stackexchange.com/a/11083/9075
\renewrobustcmd*{\bibinitdelim}{\,}

%Keep first and last name together in the bibliography
%http://tex.stackexchange.com/a/196192/9075
\renewcommand*\bibnamedelimc{\addnbspace}
\renewcommand*\bibnamedelimd{\addnbspace}

%Replace last "and" by comma in bibliography
%See http://tex.stackexchange.com/a/41532/9075
\AtBeginBibliography{%
  \renewcommand*{\finalnamedelim}{\addcomma\space}%
}

\DefineBibliographyStrings{ngerman}{
  backrefpage  = {zitiert auf S\adddot},
  backrefpages = {zitiert auf S\adddot},
  andothers    = {et\ \addabbrvspace al\adddot},
  %Tipp von http://www.mrunix.de/forums/showthread.php?64665-biblatex-Kann-%DCberschrift-vom-Inhaltsverzeichnis-nicht-%E4ndern&p=293656&viewfull=1#post293656
  bibliography = {Literaturverzeichnis}
}

%enable hyperlinked author names when using \citeauthor
%source: http://tex.stackexchange.com/a/75916/9075
\DeclareCiteCommand{\citeauthor}
  {\boolfalse{citetracker}%
   \boolfalse{pagetracker}%
   \usebibmacro{prenote}}
  {\ifciteindex
     {\indexnames{labelname}}
     {}%
   \printtext[bibhyperref]{\printnames{labelname}}}
  {\multicitedelim}
  {\usebibmacro{postnote}}

%natbib compatibility
%\newcommand{\citep}[1]{\cite{#1}}
%\newcommand{\citet}[1]{\citeauthor{#1} \cite{#1}}
%Beginning of sentence - analogous to cleveref - important for names such as "zur Muehlen"
%\newcommand{\Citep}[1]{\cite{#1}}
%\newcommand{\Citet}[1]{\Citeauthor{#1} \cite{#1}}
%%%


%%%
% Blindtext. Paket "blindtext" ist fortgeschritterner als "lipsum" und kann auch Mathematik im Text (http://texblog.org/2011/02/26/generating-dummy-textblindtext-with-latex-for-testing/)
% kantlipsum (https://www.ctan.org/tex-archive/macros/latex/contrib/kantlipsum) ist auch ganz nett, aber eben auch keine Mathematik
% Wird verwendet, um etwas Text zu erzeugen, um eine volle Seite wegen Layout zu sehen.
\usepackage[math]{blindtext}
%%%

%%%
% Neue Pakete bitte VOR hyperref einbinden. Insbesondere bei Verwendung des
% Pakets "index" wichtig, da sonst die Referenzierung nicht funktioniert.
% Für die Indizierung selbst ist unter http://xindy.sourceforge.net
% ein gutes Tool zu erhalten
%%%


%%%
%
% hier also neue packages einbinden
%
%%%


%%%
% ggf.in der Endversion komplett rausnehmen. dann auch \href in commands.tex aktivieren
% Alle Optionen nach \hypersetup verschoben, sonst crash
%
\usepackage[]{hyperref}%siehe auch: "Praktisches LaTeX" - www.itp.uni-hannover.de/~kreutzm
%
%% Da es mit KOMA 3 und xcolor zu Problemen mit den global Options kommt MÜSSEN die Optionen so gesetzt werden.
%

% Eigene Farbdefinitionen ohne die Namen des xcolor packages
\definecolor{darkblue}{rgb}{0,0,.5}
\definecolor{black}{rgb}{0,0,0}

\hypersetup{
    breaklinks=true,
    bookmarksnumbered=true,
    bookmarksopen=true,
    bookmarksopenlevel=1,
    breaklinks=true,
    colorlinks=true,
    pdfstartview=Fit,
    pdfpagelayout=TwoPageRight, % zweiseitige Darstellung: ungerade Seiten rechts im PDF-Viewer - siehe auch http://tex.stackexchange.com/a/21109/9075
    filecolor=darkblue,
    urlcolor=darkblue,
    linkcolor=black,
    citecolor=black
}
%
%%%


%%%
% cleveref für cref statt autoref, da cleveref auch bei Definitionen funktioniert
\ifdeutsch
\usepackage[ngerman,capitalise,nameinlink,noabbrev]{cleveref}
\else
\usepackage[capitalise,nameinlink,noabbrev]{cleveref}
\fi
%%%


%%%
% Zur Darstellung von Algorithmen
% Algorithm muss nach hyperref geladen werden
\usepackage[chapter]{algorithm}
\usepackage[]{algpseudocode}
%
%%%


%%%
% Schriften
%%%
%
\automark[section]{chapter}
\ifenglisch
%serif font also in heading, foot and page number (contained in foot)
\setkomafont{pageheadfoot}{\normalfont\rmfamily}
\setkomafont{pagenumber}{\normalfont\rmfamily}
\else
%sans serif font in German texts
\setkomafont{pageheadfoot}{\normalfont\sffamily}
\setkomafont{pagenumber}{\normalfont\sffamily}
\fi
%
%\setheadsepline[.4pt]{.4pt} %funktioniert nicht: Alle Linien sind hier weg
%
%%%

%%%
%
\ifenglisch
% Fuer englische Texte sind serifenhafte Ueberschriften gut. Deshalb hier der Befehl zum Aktivieren von serifenhaften Ueberschriften
\setkomafont{disposition}{\normalfont\rmfamily}

% Bei englischen Texten das Label (optionaler Eintrag bei \item) bei description-Umgegungen nur auf fett und nicht fett+serifenlos stellen.
\setkomafont{descriptionlabel}{\normalfont\bfseries}
\fi
%
%%%

%%%
% Fuer deutsche Texte: Weniger Silbentrennung, mehr Abstand zwischen den Woertern
\ifdeutsch
\setlength{\emergencystretch}{3em} % Silbentrennung reduzieren durch mehr frei Raum zwischen den Worten
\fi
%%%

%Symbole
%--------
%\usepackage[geometry]{ifsym} % \BigSquare
%\usepackage{mathabx}
%\usepackage{stmaryrd} %fuer \ovee, \owedge, \otimes
%\usepackage{marvosym} %fuer \Writinghand %patched to not redefine \Rightarrow
%\usepackage{mathrsfs} %mittels \mathscr{} schoenen geschwungenen Buchstaben erzeugen
%\usepackage{calrsfs} %\mathcal{} ein bisserl dickeren buchstaben erzeugen - sieht net so gut aus.
                      %durch mathpazo ist das schon definiert
\usepackage{amssymb}

%For \texttrademark{}
\usepackage{textcomp}

%name-clashes von marvosym und mathabx vermeiden:
\def\delsym#1{%
%  \expandafter\let\expandafter\origsym\expandafter=\csname#1\endcsname
%  \expandafter\let\csname orig#1\endcsname=\origsym
  \expandafter\let\csname#1\endcsname=\relax
}

%\usepackage{pifont}
%\usepackage{bbding}
%\delsym{Asterisk}
%\delsym{Sun}\delsym{Mercury}\delsym{Venus}\delsym{Earth}\delsym{Mars}
%\delsym{Jupiter}\delsym{Saturn}\delsym{Uranus}\delsym{Neptune}
%\delsym{Pluto}\delsym{Aries}\delsym{Taurus}\delsym{Gemini}
%\delsym{Rightarrow}
%\usepackage{mathabx} - Ueberschreibt leider zu viel - und die \le-Zeichen usw. sehen nicht gut aus!


%Fallback-Schriftart
\usepackage{lmodern}  % Latin Modern Fonts sind die Nachfolger von Computer Modern, den LaTeX-Standardfonts
%Quelle: http://homepage.ruhr-uni-bochum.de/Georg.Verweyen/pakete.html
%Allerdings sieht diese Schritart in Diplomarbeiten fuer Fliesstext auch nicht besonders schoen aus.
%Trotzdem ist sie fuer Programmcode gut geeignet

%Schriftart fuer die Ueberschriften - ueberschreibt lmodern
\ifdeutsch
\usepackage[scaled=.95]{helvet}
\else
\usepackage[scaled=.90]{helvet}
\fi

% Für Schreibschrift würde tun, muss aber ned
%\usepackage{mathrsfs} %  \mathscr{ABC}

%Schriftart fuer den Fliesstext - ueberschreibt lmodern
%
\ifdeutsch
%
%Linux Libertine, siehe http://www.linuxlibertine.org/
%Packageparamter [osf] = Minuskel-Ziffern
%rm = libertine im Brottext, Linux Biolinum NICHT als serifenlose Schrift, sondern helvet (von oben) beibehalten
\usepackage[rm]{libertine}
%
%Alternative Schriftart: Palantino, Packageparamter [osf] = Minuskel-Ziffern
%\usepackage{mathpazo} %ftp://ftp.dante.de/tex-archive/fonts/mathpazo/ - Tipp aus DE-TEX-FAQ 8.2.1
%
\fi

\ifenglisch
%
\usepackage{charter} %Charter fuer englische Texte
\linespread{1.05} % Durchschuss für Charter leicht erhöhen
%
%\usepackage{mathptmx} %Times fuer englische Texte. Sieht nicht sooo gut aus.
%
%Fallback ist lmodern, die oben eingebunden wurde
\fi

%Schriftart fuer Programmcode - ueberschreibt lmodern
%Falls auskommentiert, wird die Standardschriftart lmodern genommen
%\usepackage[scaled=.92]{luximono} % Fuer schreibmaschinenartige Schluesselwoerter in den Listings - geht bei alten Installationen nicht, da einige Fontshapes (<>=) fehlen
%\usepackage{courier}
\usepackage[scaled=0.83]{beramono} %BeraMono als Typewriter-Schrift, Tipp von http://tex.stackexchange.com/a/71346/9075

\ifluatex
\else
\usepackage[T1]{fontenc}
\fi


% optischer Randausgleich - bei miktex gleich dabei - bei linux von
%  http://www.ctan.org/tex-archive/macros/latex/contrib/microtype/
%  herunterladen 
\usepackage{microtype}
%Falls bei einer Silbentrennung ploetzlich eine ganze Zeile fehlt (passiert unter Windows XP mit MikTex 2.5 und foxit reader als pdfreader
%\usepackage{pdfcprot}
%ausprobieren. Dieses erzeugt allerdings nur für Palatino (in dieser Vorlage die Default-Schrift) einen guten optischen Randausgleich
%Falls alle Stricke reissen, muss leider auf den optischen Randausgleich verzichtet werden.

%fuer microtype
%tracking=true muss als Parameter des microtype-packages mitgegeben werden
%
%Deaktiviert, da dies bei Algorithmen seltsam aussieht
%
%\DeclareMicrotypeSet*[tracking]{my}{ font = */*/*/sc/* }% 
%\SetTracking{ encoding = *, shape = sc }{ 45 }% Hier wird festgelegt,
            % dass alle Passagen in Kapitälchen automatisch leicht
            % gesperrt werden.
			% Quelle: http://homepage.ruhr-uni-bochum.de/Georg.Verweyen/pakete.html

%
%%%


%%%
% Links auf Gleitumgebungen springen nicht zur Beschriftung,
% Doc: http://mirror.ctan.org/tex-archive/macros/latex/contrib/oberdiek/hypcap.pdf
% sondern zum Anfang der Gleitumgebung
\usepackage[all]{hypcap}
%%%


%%%
% Deckblattstyle
%
\ifdeutsch
	\PassOptionsToPackage{language=german}{uni-stuttgart-cs-cover}
\else
	\PassOptionsToPackage{language=english}{uni-stuttgart-cs-cover}
\fi

\usepackage[
    title={Förderungswürdigkeit der F\"{o}rderung von Öl},
    author={Lars K.},
    type=bachelor,
    institute=iaas,
    number=12345, % IF you do not have a number and do not need one leave the number field blank e.g. number=,
    course=se,
    examiner={Prof.\ Dr.\ Uwe Fessor},
    supervisor={Dipl.-Inf.\ Roman Tiker,\\Dipl.-Inf.\ Laura Stern,\\Otto Normalverbraucher,\ M.Sc.},
    startdate={5.\ Juli 2013}, % English: July 5, 2013;    ISO: 2013-07-05
    enddate={5.\ Januar 2014}, % English: January 5, 2014; ISO: 2014-01-05
    crk={I.7.2}
    ]{uni-stuttgart-cs-cover}
%
%%%


%%%
%Bugfixes packages
%\usepackage{fixltx2e} %Fuer neueste LaTeX-Installationen nicht mehr benoetigt - bereinigte einige Ungereimtheiten, die auf Grund von Rueckwaertskompatibilitaet beibahlten wurden.
%\usepackage{mparhack} %Fixt die Position von marginpars (die in DAs selten bis gar nicht gebraucht werden}
%\usepackage{ellipsis} %Fixt die Abstaende vor \ldots. Wird wohl auch nicht benoetigt.
%
%%%


%%%
% Rand
%Viele Moeglichkeiten, die Raender im Dokument einzustellen.
%Satzspiegel neu berechnen. Dokumentation dazu ist in "scrguide.pdf" von KOMA-Skript zu finden
%  Optionen werden bei \documentclass[] in ausarbeitung.tex mitgegeben.
\typearea[current]{current} %neu berechnen, da neue Schrift eingebunden

%\usepackage{a4}
%\usepackage{a4wide}
%\areaset{170mm}{277mm} %a4:29,7hochx21mbreit

%Wer die Masse direkt eingeben moechte:
%Bei diesem Beispiel wird die Regel nicht beachtet, dass der innere Rand halb so gross wie der aussere Rand und der obere Rand halb so gross wie der untere Rand sein sollte
%\usepackage[inner=2.5cm, outer=2.5cm, includefoot, top=3cm, bottom=1.5cm]{geometry}



%
%%%


%%%
% Optionen
%
\captionsetup{
  format=hang,
  labelfont=bf,
  justification=justified,
  %single line captions should be centered, multiline captions justified
  singlelinecheck=true
}
%
%neue float Umgebung fuer Listings, die mittels fancyvrb gesetzt werden sollen
\floatstyle{ruled}
\newfloat{Listing}{tbp}{code}[chapter]
\crefname{Listing}{Listing}{Listings}
\newfloat{Algorithmus}{tbp}{alg}[chapter]
\ifdeutsch
\crefname{Algorithmus}{Algorithmus}{Algorithmus}
\else
\crefname{Algorithmus}{Algorithm}{Algorithms}
\fi
%
%amsmath
%\numberwithin{equation}{section}
%\renewcommand{\theequation}{\thesection.\Roman{equation}}
%
%pdftex
\pdfcompresslevel=9
%
%Tabellen (array.sty)
\setlength{\extrarowheight}{1pt}
%
%
%%%

%%%
% unterschiedliche Chapter-Styles
% u.a. Paket fncychap

% Andere Kapitelueberschriften
% falls einem der Standard von KOMA nicht gefaellt...
% Falls man zurück zu KOMA moechte, dann muss jede der vier folgenden Moeglichkeiten deaktiviert sein.

% 1. Moeglichkeit
%\usepackage[Sonny]{fncychap}
%oder
%\usepackage[Bjarne]{fncychap}
%oder
%\usepackage[Lenny]{fncychap}

% 2. Moeglichkeit
\iffalse
\usepackage[Bjarne]{fncychap}
\ChNameVar{\Large\sf} \ChNumVar{\Huge} \ChTitleVar{\Large\sf}
\ChRuleWidth{0.5pt} \ChNameUpperCase
\fi

%Variante der 2. Moeglichkeit
\iffalse
\usepackage[Rejne]{fncychap}
\ChNameVar{\centering\Huge\rm\bfseries}
\ChNumVar{\Huge}
 \ChTitleVar{\centering\Huge\rm}
\ChNameUpperCase
\ChTitleUpperCase
\ChRuleWidth{1pt}
\fi

% 3. Moeglichkeit
\iffalse
\usepackage{fncychap}
\ChNameUpperCase
\ChTitleUpperCase
\ChNameVar{\raggedright\normalsize} %\rm
\ChNumVar{\bfseries\Large}
\ChTitleVar{\raggedright\Huge}
\ChRuleWidth{1pt}
\fi

% 4. Moeglichkeit
% Zur Aktivierierung "\iffalse" und "\fi" auskommentieren
% Innen drin kann man dann noch zwischen
%   * serifenloser Schriftart (eingestellt)
%   * serifenhafter Schriftart (wenn kein zusaetzliches Kommando aktiviert ist) und
%   * Kapitälchen wählen
\iffalse
\makeatletter
%\def\thickhrulefill{\leavevmode \leaders \hrule height 1ex \hfill \kern \z@}

%Fuer Kapitel mit Kapitelnummer
\def\@makechapterhead#1{%
  \vspace*{10\p@}%
  {\parindent \z@ \raggedright \reset@font
			%Default-Schrift: Serifenhaft (gut fuer englische Dokumente)
            %A) Fuer serifenlose Schrift:
            \fontfamily{phv}\selectfont
			%B) Fuer Kapitaelchen:
			%\fontseries{m}\fontshape{sc}\selectfont
            %C) Fuer ganz "normale" Schrift:
            %\normalfont 
			%
			\Large \@chapapp{} \thechapter
        \par\nobreak\vspace*{10\p@}%
        \interlinepenalty\@M
    {\Huge\bfseries\baselineskip3ex
	%Fuer Kapitaelchen folgende Zeile aktivieren:
	%\fontseries{m}\fontshape{sc}\selectfont
	#1\par\nobreak}
    \vspace*{10\p@}%
\makebox[\textwidth]{\hrulefill}%    \hrulefill alone does not work
    \par\nobreak
    \vskip 40\p@
  }}

  %Fuer Kapitel ohne Kapitelnummer (z.B. Inhaltsverzeichnis)
  \def\@makeschapterhead#1{%
  \vspace*{10\p@}%
  {\parindent \z@ \raggedright \reset@font
            \normalfont \vphantom{\@chapapp{} \thechapter}
        \par\nobreak\vspace*{10\p@}%
        \interlinepenalty\@M
    {\Huge \bfseries %
	%Default-Schrift: Serifenhaft (gut fuer englische Dokumente)
    %A) Fuer serifenlose Schrift folgende Zeile aktivieren:
    \fontfamily{phv}\selectfont
	%B) Fuer Kapitaelchen folgende Zeile aktivieren:
	%\fontseries{m}\fontshape{sc}\selectfont
	#1\par\nobreak}
    \vspace*{10\p@}%
\makebox[\textwidth]{\hrulefill}%    \hrulefill does not work
    \par\nobreak
    \vskip 40\p@
  }}
%
\makeatother
\fi

%%%

%%%
%Minitoc-Einstellungen
%\dominitoc
%\renewcommand{\mtctitle}{Inhaltsverzeichnis dieses Kapitels}
%
% Disable single lines at the start of a paragraph (Schusterjungen)
\clubpenalty = 10000
%
% Disable single lines at the end of a paragraph (Hurenkinder)
\widowpenalty = 10000 \displaywidowpenalty = 10000
%
%http://groups.google.de/group/de.comp.text.tex/browse_thread/thread/f97da71d90442816/f5da290593fd647e?lnk=st&q=tolerance+emergencystretch&rnum=5&hl=de#f5da290593fd647e
%Mehr Infos unter http://www.tex.ac.uk/cgi-bin/texfaq2html?label=overfull
\tolerance=2000
\setlength{\emergencystretch}{3pt}   % kann man evtl. auf 20 erhoehen
\setlength{\hfuzz}{1pt}
%
%%%


%%%
% Fuer listings.sty
\lstset{language=XML,
        showstringspaces=false,
        extendedchars=true,
        basicstyle=\footnotesize\ttfamily,
        commentstyle=\slshape,
        stringstyle=\ttfamily, %Original: \rmfamily, damit werden die Strings im Quellcode hervorgehoben. Zusaetzlich evtl.: \scshape oder \rmfamily durch \ttfamily ersetzen. Dann sieht's aus, wie bei fancyvrb
        breaklines=true,
        breakatwhitespace=true,
        columns=flexible,
        aboveskip=0mm, %deaktivieren, falls man lstlistings direkt als floating object benutzt (\begin{lstlisting}[float,...])
        belowskip=0mm, %deaktivieren, falls man lstlistings direkt als floating object benutzt (\begin{lstlisting}[float,...])
        captionpos=b
}
\ifdeutsch
\renewcommand{\lstlistlistingname}{Verzeichnis der Listings}
\fi
%
%%%


%%%
%fuer algorithm.sty: - falls Deutsch und nicht Englisch. Falls Englisch als Sprache gewählt wurde, bitte die folgenden beiden Zeilen auskommentieren.
\floatname{algorithm}{Algorithmus}
\ifdeutsch
\renewcommand{\listalgorithmname}{Verzeichnis der Algorithmen}
\fi
%
%%%


%%%
% Das Euro Zeichen
% Fuer Palatino (mathpazo.sty): richtiges Euro-Zeichen
% Alternative: \usepackage{eurosym}
\newcommand{\EUR}{\ppleuro}
%
%%%


%%%
%
% Float-placements - http://dcwww.camd.dtu.dk/~schiotz/comp/LatexTips/LatexTips.html#figplacement
% and http://people.cs.uu.nl/piet/floats/node1.html
\renewcommand{\topfraction}{0.85}
\renewcommand{\bottomfraction}{0.95}
\renewcommand{\textfraction}{0.1}
\renewcommand{\floatpagefraction}{0.75}
%\setcounter{totalnumber}{5}
%
%%%

%%%
%
% Bei Gleichungen nur dann die Nummer zeigen, wenn die Gleichung auch referenziert wird
%
% Funktioniert mit MiKTeX Stand 2012-01-13 nicht. Deshalb ist dieser Schalter deaktiviert.
%
%\mathtoolsset{showonlyrefs}
%
%%%


%%%
%ensure that floats covering a whole page are placed at the top of the page
%see http://tex.stackexchange.com/a/28565/9075
\makeatletter
\setlength{\@fptop}{0pt}
\setlength{\@fpbot}{0pt plus 1fil}
\makeatother
%%%


%%%
%Optischer Randausgleich
\usepackage{microtype}
%%%

%%%
%Package geometry to enlarge on page
%
%Source: http://www.howtotex.com/tips-tricks/change-margins-of-a-single-page/
%
%Normally, this should not be used as the typearea package calculates the margins perfectly
\usepackage[
  pass %just load the package and do not destory the work of typearea
]{geometry}
%%%

%%%
% footnotes in tables
\usepackage{footnote}
\makesavenoteenv{tabular}
\makesavenoteenv{table}
% Reuse of footnotes
% Reuse of Footnotes, see http://tex.stackexchange.com/questions/10102/multiple-references-to-the-same-footnote-with-hyperref-support-is-there-a-bett
\crefformat{footnote}{#2\footnotemark[#1]#3}
%%%

%%%
% pgfplots (optional if the ppackage is installed)
% PGFPlots draws high-qual­ity func­tion plots in nor­mal or log­a­rith­mic scal­ing
\IfFileExists{pgfplots.sty}{
\usepackage{pgfplots}
\pgfplotsset{compat=1.12}
}{}
%%%

%%%
% tikz (optional if the ppackage is installed)
% Package for creating graphics programmatically
\IfFileExists{tikz.sty}{
\usepackage{tikz}
}{}
%%%

%Der untere Rand darf "flattern"
\raggedbottom

%%%
% Wie tief wird das Inhaltsverzeichnis aufgeschlüsselt
% 0 --\chapter
% 1 --\section % fuer kuerzeres Inhaltsverzeichnis verwenden - oder minitoc benutzen
% 2 --\subsection
% 3 --\subsubsection
% 4 --\paragraph
\setcounter{tocdepth}{1}
%
%%%

\makeindex

%Angaben in die PDF-Infos uebernehmen
\makeatletter
\hypersetup{
            pdftitle={}, %Titel der Arbeit
            pdfauthor={}, %Author
            pdfkeywords={}, % CR-Klassifikation und ggf. weitere Stichworte
            pdfsubject={}
}
\makeatother

% Hier stehen alle Abkürzungen
\newacronym{er}{ER}{error rate}
\newacronym{fr}{FR}{Fehlerrate}
\newacronym[plural={RDBMS},shortplural={RDBMS}]{rdbms}{RDBMS}{Relational Database Management System}

\begin{document}

%tex4ht-Konvertierung verschönern
\iftex4ht
% tell tex4ht to create picures also for formulas starting with '$$'
% WARNING: a tex4ht run now takes forever!
\Configure{$$}{\PicMath}{\EndPicMath}{}
%$$ % <- syntax highlighting fix for emacs
\Css{body {text-align:justify;}}

%conversion of .pdf to .png
\Configure{graphics*}
         {pdf}
         {\Needs{"convert \csname Gin@base\endcsname.pdf
                               \csname Gin@base\endcsname.png"}%
          \Picture[pict]{\csname Gin@base\endcsname.png}%
         }
\fi

%Tipp von http://goemonx.blogspot.de/2012/01/pdflatex-ligaturen-und-copynpaste.html
%siehe auch http://tex.stackexchange.com/questions/4397/make-ligatures-in-linux-libertine-copyable-and-searchable
%
%ONLY WORKS ON MiKTeX
%On other systems, download glyphtounicode.tex from http://pdftex.sarovar.org/misc/
%
\input glyphtounicode.tex
\pdfgentounicode=1

%\VerbatimFootnotes %verbatim text in Fußnoten erlauben. Geht normalerweise nicht.

%wird fuer Tabellen benötigt (z.B. >{centering\RBS}p{2.5cm} erzeugt einen zentrierten 2,5cm breiten Absatz in einer Tabelle
\newcommand{\RBS}{\let\\=\tabularnewline}

%To avoid issues with Springer's \mathplus
%See also http://tex.stackexchange.com/q/212644/9075
\providecommand\mathplus{+}

%% typoraphisch richtige Abkürzungen
\newcommand{\zB}[0]{z.\,B.\xspace}
\newcommand{\bzw}[0]{bzw.\xspace}
\newcommand{\usw}[0]{usw.\xspace}
\renewcommand{\dh}[0]{d.\,h.\xspace}

%from hmks makros.tex - \indexify
\newcommand{\toindex}[1]{\index{#1}#1}
%
\newcommand{\dotcup}{\ensuremath{\,\mathaccent\cdot\cup\,}} %Tipp aus The Comprehensive LaTeX Symbol List
%
%Anstatt $|x|$ $\abs{x}$ verwenden. Die Betragsstriche skalieren automatisch, falls "x" etwas größer sein sollte...
\newcommand{\abs}[1]{\left\lvert#1\right\rvert}
%
%für Zitate
\newcommand{\citeS}[2]{\cite[S.~#1]{#2}}
\newcommand{\citeSf}[2]{\cite[S.~#1\,f.]{#2}}
\newcommand{\citeSff}[2]{\cite[S.~#1\,ff.]{#2}}
\newcommand{\vgl}{vgl.\ }
\newcommand{\Vgl}{Vgl.\ }
%
\newcommand{\commentchar}{\ensuremath{/\mkern-4mu/}}
\algrenewcommand{\algorithmiccomment}[1]{\hfill $$\commentchar$$ #1}

% Seitengrößen - Gegen Schusterjungen und Hurenkinder...
\newcommand{\largepage}{\enlargethispage{\baselineskip}}
\newcommand{\shortpage}{\enlargethispage{-\baselineskip}}

\pagenumbering{arabic}
\Titelblatt

%Eigener Seitenstil fuer die Kurzfassung und das Inhaltsverzeichnis
\deftripstyle{preamble}{}{}{}{}{}{\pagemark}
%Doku zu deftripstyle: scrguide.pdf
\pagestyle{preamble}
\renewcommand*{\chapterpagestyle}{preamble}

%Kurzfassung / abstract
%auch im Stil vom Inhaltsverzeichnis
\ifdeutsch
\section*{Kurzfassung}
\else
\section*{Abstract}
\fi
\ldots ... Short summary of the thesis ...
\cleardoublepage


% BEGIN: Verzeichnisse

\iftex4ht
\else
\microtypesetup{protrusion=false}
\fi

%%%
% Literaturverzeichnis ins TOC mit aufnehmen, aber nur wenn nichts anderes mehr hilft!
% \addcontentsline{toc}{chapter}{Literaturverzeichnis}
%
% oder zB
%\addcontentsline{toc}{section}{Abkürzungsverzeichnis}
%
%%%

%Produce table of contents
%
%In case you have trouble with headings reaching into the page numbers, enable the following three lines.
%Hint by http://golatex.de/inhaltsverzeichnis-schreibt-ueber-rand-t3106.html
%
%\makeatletter
%\renewcommand{\@pnumwidth}{2em}
%\makeatother
%
\tableofcontents

% Bei einem ungünstigen Seitenumbruch im Inhaltsverzeichnis, kann dieser mit
% \addtocontents{toc}{\protect\newpage}
% an der passenden Stelle im Fließtext erzwungen werden.

\listoffigures
\listoftables

%Wird nur bei Verwendung von der lstlisting-Umgebung mit dem "caption"-Parameter benoetigt
%\lstlistoflistings
%ansonsten:
\ifdeutsch
\listof{Listing}{Verzeichnis der Listings}
\else
\listof{Listing}{List of Listings}
\fi

%mittels \newfloat wurde die Algorithmus-Gleitumgebung definiert.
%Mit folgendem Befehl werden alle floats dieses Typs ausgegeben
\ifdeutsch
\listof{Algorithmus}{Verzeichnis der Algorithmen}
\else
\listof{Algorithmus}{List of Algorithms}
\fi
%\listofalgorithms %Ist nur für Algorithmen, die mittels \begin{algorithm} umschlossen werden, nötig

% Abkürzungsverzeichnis
\printnoidxglossaries

\iftex4ht
\else
%Optischen Randausgleich und Grauwertkorrektur wieder aktivieren
\microtypesetup{protrusion=true}
\fi

% END: Verzeichnisse


\renewcommand*{\chapterpagestyle}{scrplain}
\pagestyle{scrheadings}
\pagestyle{scrheadings}

%ihead aufgeteilt - Bezeichnungen: 4.1, S. 119, scrguide

%für die Teilversionen - nur bei Verwendung von RCS/CVS
%\ihead[Version \RCSRevision]{Version \RCSRevision}

%Für die finale Version oder bei Verwendung von SVN
\ihead[]{}


% Sowohl für die Teilversionen als auch die finale Version:

\chead[]{}
\ohead[]{\headmark}
%
\cfoot[]{}
\ofoot[\usekomafont{pagenumber}\thepage]{\usekomafont{pagenumber}\thepage}
\ifoot[]{}

%
%
% ** Hier wird der Text eingebunden **
%
% !TeX spellcheck = de_DE

\chapter{Einleitung}
In diesem Kapitel steht die Einleitung zu dieser Arbeit.
Sie soll nur als Beispiel dienen und hat nichts mit dem Buch \cite{WSPA} zu tun.
Nun viel Erfolg bei der Arbeit!

Bei \LaTeX\ werden Absätze durch freie Zeilen angegeben.
Da die Arbeit über ein Versionskontrollsystem versioniert wird, ist es sinnvoll, pro \emph{Satz} eine neue Zeile im \texttt{.tex}-Dokument anzufangen.
So kann einfacher ein Vergleich von Versionsständen vorgenommen werden.

\section*{Gliederung}
Die Arbeit ist in folgender Weise gegliedert:
\begin{description}
\item[Kapitel~\ref{chap:k2} -- \nameref{chap:k2}:] Hier werden werden die Grundlagen dieser Arbeit beschrieben.
\item[Kapitel~\ref{chap:zusfas} -- \nameref{chap:zusfas}] fasst die Ergebnisse der Arbeit zusammen und stellt Anknüpfungspunkte vor.
\end{description}

%\input{...weitere Kapitel...}
% !TeX spellcheck = de_DE

\chapter{Kapitel zwei}
\label{chap:k2}

Hier wird der Hauptteil stehen. Falls mehrere Kapitel gewünscht, entweder mehrmals \texttt{\textbackslash{}chapter} benutzen oder pro Kapitel eine eigene Datei anlegen und \texttt{ausarbeitung.tex} anpassen.

LaTeX-Hinweise stehen in \cref{chap:latextipps}.

Look here: $document.type$

%noch etwas Fülltext
\blinddocument

% !TeX spellcheck = de_DE

\chapter{Zusammenfassung und Ausblick}\label{chap:zusfas}
Hier bitte einen kurzen Durchgang durch die Arbeit.

\section*{Ausblick}
...und anschließend einen Ausblick


%
%
%\renewcommand{\appendixtocname}{Anhang}
%\renewcommand{\appendixname}{Anhang}
%\renewcommand{\appendixpagename}{Anhang}
\appendix
% !TeX spellcheck = de_DE
%Die Angabe des schlauen Spruchs auf diesem Wege funtioniert nur,
%wenn keine Änderung des Kapitels mittels den in preambel/chapterheads.tex
%vorgeschlagenen Möglichkeiten durchgeführt wurde.
\setchapterpreamble[u]{%
\dictum[Albert Einstein]{Probleme kann man niemals mit derselben Denkweise lösen, durch die sie entstanden sind.}
}
\chapter{LaTeX-Tipps}
\label{chap:latextipps}

Pro Satz eine neue Zeile.
Das ist wichtig, um sauber versionieren zu können.
In LaTeX werden Absätze durch eine Leerzeile getrennt.

Folglich werden neue Abstäze insbesondere \emph{nicht} durch Doppelbackslashes erzeugt.
Der letzte Satz kam in einem neuen Absatz.

\section{File-Encoding und Unterstützung von Umlauten}
\label{sec:firstsectioninlatexhints}
Die Vorlage wurde 2010 auf UTF-8 umgestellt.
Alle neueren Editoren sollten damit keine Schwierigkeiten haben.

\section{Zitate}
Referenzen werden mittels \texttt{\textbackslash cite[key]} gesetzt.
Beispiel: \cite{WSPA} oder mit Autorenangabe: \citet{WSPA}.

Der folgende Satz demonstriert \begin{inparaenum}[1.]
\item die Großschreibung von Autorennamen am Satzanfang,
\item die richtige Zitation unter Verwendung von Autorennamen und der Referenz,
\item dass die Autorennamen ein Hyperlink auf das Literaturverzeichnis sind sowie
\item dass in dem Literaturverzeichnis der Namenspräfix \enquote{van der} von \enquote{Wil M.\,P.\ van der Aalst} steht.
\end{inparaenum}
\Citet{RVvdA2016} präsentieren eine Studie über die Effektivität von Workflow-Management-Systemen.

Der folgende Satz demonstriert, dass man mittels \texttt{label} in einem Bibliopgrahie"=Eintrag den Textteil des generierten Labels überschreiben kann, aber das Jahr und die Eindeutigkeit noch von biber generiert wird.
Die Apache ODE Engine \cite{ApacheODE} ist eine Workflow-Maschine, die BPEL-Prozesse zuverlässig ausführt.

Wörter am besten mittels \texttt{\textbackslash enquote\{...\}} \enquote{einschließen}, dann werden die richtigen Anführungszeichen verwendet.

Beim Erstellen der Bibtex-Datei wird empfohlen darauf zu achten, dass die DOI aufgeführt wird.

\section{Mathematische Formeln}
\label{sec:mf}
Mathematische Formeln kann man $so$ setzen. \texttt{symbols-a4.pdf} (zu finden auf \url{http://www.ctan.org/tex-archive/info/symbols/comprehensive/symbols-a4.pdf}) enthält eine Liste der unter LaTeX direkt verfügbaren Symbole.
Z.\,B.\ $\mathbb{N}$ für die Menge der natürlichen Zahlen.
Für eine vollständige Dokumentation für mathematischen Formelsatz sollte die Dokumentation zu \texttt{amsmath}, \url{ftp://ftp.ams.org/pub/tex/doc/amsmath/} gelesen werden.

Folgende Gleichung erhält keine Nummer, da \texttt{\textbackslash equation*} verwendet wurde.
\begin{equation*}
x = y
\end{equation*}

Die Gleichung~\ref{eq:test} erhält eine Nummer:
\begin{equation}
\label{eq:test}
x = y
\end{equation}

Eine ausführliche Anleitung zum Mathematikmodus von LaTeX findet sich in \url{http://www.ctan.org/tex-archive/help/Catalogue/entries/voss-mathmode.html}.

\section{Quellcode}
\Cref{lst:ListingANDlstlisting} zeigt, wie man Programmlistings einbindet.
Mittels \texttt{\textbackslash lstinputlisting} kann man den Inhalt direkt aus Dateien lesen.

%Listing-Umgebung wurde durch \newfloat{Listing} definiert
\begin{Listing}
\begin{lstlisting}
<listing name="second sample">
  <content>not interesting</content>
</listing>
\end{lstlisting}
\caption{lstlisting in einer Listings-Umgebung, damit das Listing durch Balken abgetrennt ist}
\label{lst:ListingANDlstlisting}
\end{Listing}

Quellcode im \lstinline|<listing />| ist auch möglich.

\section{Abbildungen}

Die \cref{fig:chor1} und \ref{fig:chor2} sind für das Verständnis dieses Dokuments wichtig.
Im Anhang zeigt \vref{fig:AnhangsChor} erneut die komplette Choreographie.

%Die Parameter in eckigen Klammern sind optionale Parameter - z.B. [htb!]
%htb! bedeutet: "Liebes LaTeX, bitte platziere diese Abbildung zuerst hier ("_h_ere"). Falls das nicht funktioniert, dann bitte oben auf der Seite ("_t_op"). Und falls das nicht geht, bitte unten auf der Seite ("_b_ottom"). Und bitte, bitte bevorzuge hier und oben, auch wenn's net so optimal aussieht ("!")
%Diese sollten nach Möglichkeit NICHT verwendet werden. LaTeX's Algorithmus für das Platzieren der Gleitumgebung ist schon sehr gut!
\begin{figure}
  \centering
  \includegraphics[width=\textwidth]{choreography.pdf}
  \caption{Beispiel-Choreographie}
  \label{fig:chor1}
\end{figure}

\begin{figure}
  \centering
  \includegraphics[width=.8\textwidth]{choreography.pdf}
  \caption[Beispiel-Choreographie]{Die Beispiel-Choreographie. Nun etwas kleiner, damit \texttt{\textbackslash textwidth} demonstriert wird. Und auch die Verwendung von alternativen Bildunterschriften für das Verzeichnis der Abbildungen. Letzteres ist allerdings nur Bedingt zu empfehlen, denn wer liest schon so viel Text unter einem Bild? Oder ist es einfach nur Stilsache?}
  \label{fig:chor2}
\end{figure}


\begin{figure}
  \centering
    \subfloat[]{\includegraphics[width=0.3\textwidth]{choreography.pdf} \label{fig:subfigA}}
    \subfloat[]{\includegraphics[width=0.3\textwidth]{choreography.pdf} \label{fig:subfigB}}
		\subfloat[Subcaption if needed]{\includegraphics[width=0.3\textwidth]{choreography.pdf} \label{fig:subfigC}}
	\caption{Beispiel um 3 Abbildung nebeneinader zu stellen nur jedes einzeln referenzieren zu können. Abbildung~\ref{fig:subfigB}
 ist die mittlere Abbildung.}
\label{fig:subfig_example}
\end{figure}

Es ist möglich, SVGs direkt beim Kompilieren in PDF umzuwandeln.
Dies ist im Quellcode zu latex-tipps.tex beschrieben, allerdings auskommentiert.

\iffalse % <-- Das hier wegnehmen, falls inkscape im Pfad ist
Das SVG in \cref{fig:directSVG} ist direkt eingebunden, während der Text im SVG in \cref{fig:latexSVG} mittels pdflatex gesetzt ist.
Falls man die Graphiken sehen möchte, muss inkscape im PATH sein und im Tex-Quelltext \texttt{\textbackslash{}iffalse} und \texttt{\textbackslash{}iftrue} auskommentiert sein.

\begin{figure}
\centering
\includegraphics{svgexample.svg}
\caption{SVG direkt eingebunden}
\label{fig:directSVG}
\end{figure}

\begin{figure}
\centering
\def\svgwidth{.4\textwidth}
\includesvg{svgexample}
\caption{Text im SVG mittels \LaTeX{} gesetzt}
\label{fig:latexSVG}
\end{figure}
\fi % <-- Das hier wegnehmen, falls inkscape im Pfad ist

\section{Tabellen}

\cref{tab:Ergebnisse} zeigt Ergebnisse und die \cref{tab:Ergebnisse} zeigt wie numerische Daten in einer Tabelle representiert werden können.
\begin{table}
  \centering
  \begin{tabular}{ccc}
  \toprule
  \multicolumn{2}{c}{\textbf{zusammengefasst}} & \textbf{Titel} \\ \midrule
  Tabelle & wie & in \\
  \url{tabsatz.pdf}& empfohlen & gesetzt\\

  \multirow{2}{*}{Beispiel} & \multicolumn{2}{c}{ein schönes Beispiel}\\
   & \multicolumn{2}{c}{für die Verwendung von \enquote{multirow}}\\
  \bottomrule
  \end{tabular}
  \caption[Beispieltabelle]{Beispieltabelle -- siehe \url{http://www.ctan.org/tex-archive/info/german/tabsatz/}}
  \label{tab:Ergebnisse}
\end{table}

\begin{table}
	\centering
	\begin{tabular}{l *{8}{d{3.2}}}
		\toprule
						
			   & \multicolumn{2}{c}{\textbf{Parameter 1}} & \multicolumn{2}{c}{\textbf{Parameter 2}} & \multicolumn{2}{c}{\textbf{Parameter 3}} & \multicolumn{2}{c}{\textbf{Parameter 4}} \\
			\cmidrule(r){2-3}\cmidrule(lr){4-5}\cmidrule(lr){6-7}\cmidrule(l){8-9}
			
			\textbf{Bedingungen} & \multicolumn{1}{c}{\textbf{M}} & \multicolumn{1}{c}{\textbf{SD}} & \multicolumn{1}{c}{\textbf{M}} & \multicolumn{1}{c}{\textbf{SD}} & \multicolumn{1}{c}{\textbf{M}} & \multicolumn{1}{c}{\textbf{SD}} & \multicolumn{1}{c}{\textbf{M}} & \multicolumn{1}{c}{\textbf{SD}}\\
			\midrule
			
			W & 1.1 & 5.55 & 6.66 & .01 &  &  &  & \\
			X & 22.22 & 0.0 & 77.5 & .1 &  &  &  & \\
			Y & 333.3 & .1 & 11.11 & .05 &  &  &  & \\
			Z & 4444.44 & 77.77 & 14.06 & .3 &  &  &  & \\
		\bottomrule 
	\end{tabular}
	
	\caption{Beispieltabelle f\"{u}r 4 Bedingungen (W-Z) mit jeweils 4 Parameters mit (M und SD). Hinweiß: immer die selbe anzahl an Nachkommastellen angeben.}
	\label{tab:Werte}
\end{table}

\section{Pseudocode}
\Cref{alg:sample} zeigt einen Beispielalgorithmus.
\begin{Algorithmus} %Die Umgebung nur benutzen, wenn man den Algorithmus ähnlich wie Graphiken von TeX platzieren lassen möchte
\caption{Sample algorithm}
\label{alg:sample}
\begin{algorithmic}
\Procedure{Sample}{$a$,$v_e$}
\State $\mathsf{parentHandled} \gets (a = \mathsf{process}) \lor \mathsf{visited}(a'), (a',c,a) \in \mathsf{HR}$
\State \Comment $(a',c'a) \in \mathsf{HR}$ denotes that $a'$ is the parent of $a$
\If{$\mathsf{parentHandled}\,\land(\mathcal{L}_\mathit{in}(a)=\emptyset\,\lor\,\forall l \in \mathcal{L}_\mathit{in}(a): \mathsf{visited}(l))$}
\State $\mathsf{visited}(a) \gets \text{true}$
\State $\mathsf{writes}_\circ(a,v_e) \gets
\begin{cases}
\mathsf{joinLinks}(a,v_e) & \abs{\mathcal{L}_\mathit{in}(a)} > 0\\
\mathsf{writes}_\circ(p,v_e)
& \exists p: (p,c,a) \in \mathsf{HR}\\
(\emptyset, \emptyset, \emptyset, false) & \text{otherwise}
\end{cases}
$
\If{$a\in\mathcal{A}_\mathit{basic}$}
  \State \Call{HandleBasicActivity}{$a$,$v_e$}
\ElsIf{$a\in\mathcal{A}_\mathit{flow}$}
  \State \Call{HandleFlow}{$a$,$v_e$}
\ElsIf{$a = \mathsf{process}$} \Comment Directly handle the contained activity
  \State \Call{HandleActivity}{$a'$,$v_e$}, $(a,\bot,a') \in \mathsf{HR}$
  \State $\mathsf{writes}_\bullet(a) \gets \mathsf{writes}_\bullet(a')$
\EndIf
\ForAll{$l \in \mathcal{L}_\mathit{out}(a)$}
  \State \Call{HandleLink}{$l$,$v_e$}
\EndFor
\EndIf
\EndProcedure
\end{algorithmic}
\end{Algorithmus}

\clearpage
Und wer einen Algorithmus schreiben möchte, der über mehrere Seiten geht, der kann das nur mit folgendem \textbf{üblen} Hack tun:

{
\begin{minipage}{\textwidth}
\hrule height .8pt width\textwidth
\vskip.3em%\vskip\abovecaptionskip\relax
\stepcounter{Algorithmus}
\addcontentsline{alg}{Algorithmus}{\protect\numberline{\theAlgorithmus}{\ignorespaces Description \relax}}
\noindent\textbf{Algorithmus \theAlgorithmus} Description
%\stepcounter{algorithm}
%\addcontentsline{alg}{Algorithmus}{\thealgorithm{}\hskip0em Description}
%\textbf{Algorithmus \thealgorithm} Description
\vskip.3em%\vskip\belowcaptionskip\relax
\hrule height .5pt width\textwidth
\end{minipage}
%without the following line, the text is nerer at the rule
\vskip-.3em
%
code goes here\\
test2\\
%
\vskip-.7em
\hrule height .5pt width\textwidth
}


\section{Abkürzungen}

Beim ersten Durchlauf betrug die \gls{fr} 5.
Beim zweiten Durchlauf war die \gls{fr} 3.~Die Pluralform sieht man hier:\ \glspl{er}.
Um zu demonstrieren, wie das Abkürzungsverzeichnis bei längeren Beschreibungstexten aussieht, muss hier noch \glspl{rdbms} erwähnt werden.

Mit \verb+\gls{...}+ können Abkürzungen eingebaut werden, beim ersten Aufrufen wird die lange Form eingesetzt.
Beim wiederholten Verwenden von \verb+\gls{...}+ wird automatisch die kurz Form angezeigt.
Außerdem wird die Abkürzung automatisch in die Abkürzungsliste eingefügt.
Mit \verb+\glspl{...}+ wird die Pluralform verwendet.
Möchte man, dass bei der ersten Verwendung direkt die Kurzform erscheint, so kann man mit \verb+\glsunset{...}+ eine Abkürzung als bereits verwendet markieren.
Das Gegenteil erreicht man mit \verb+\glsreset{...}+.

Definiert werden Abkürzungen in der Datei \textit{content\\ausarbeitung.tex} mithilfe von \verb+\newacronym{...}{...}{...}+.

Mehr Infos unter: \url{http://tug.ctan.org/macros/latex/contrib/glossaries/glossariesbegin.pdf}

\section{Verweise}
Für weit entfernte Abschnitte ist \enquote{varioref} zu empfehlen:
\enquote{Siehe \vref{sec:mf}}.
Das Kommando \texttt{\textbackslash{}vref} funktioniert ähnlich wie \texttt{\textbackslash{}cref} mit dem Unterschied, dass zusätzlich ein Verweis auf die Seite hinzugefügt wird.
\texttt{vref}: \enquote{\vref{sec:firstsectioninlatexhints}}, \texttt{cref}: \enquote{\cref{sec:firstsectioninlatexhints}}, \texttt{ref}: \enquote{\ref{sec:firstsectioninlatexhints}}.

Falls \enquote{varioref} Schwierigkeiten macht, dann kann man stattdessen \enquote{cref} verwenden.
Dies erzeugt auch das Wort \enquote{Abschnitt} automatisch: \cref{sec:mf}.
Das geht auch für Abbildungen usw.
Im Englischen bitte \verb1\Cref{...}1 (mit großem \enquote{C} am Anfang) verwenden.


%Mit MiKTeX Installation ab dem 2012-01-16 nicht mehr nötig
%Falls ein Abschnitt länger als eine Seite wird und man mittels \texttt{\textbackslash{}vref} auf eine konkrete Stelle in der Section
%verweisen möchte, dann sollte man \texttt{\textbackslash{}phantomsection} verwenden und dann wird
%auch bei \texttt{vref} die richtige Seite angeben.

%%The link location will be placed on the line below.
%%Tipp von http://en.wikibooks.org/wiki/LaTeX/Labels_and_Cross-referencing#The_hyperref_package_and_.5Cphantomsection
%\phantomsection
%\label{alabel}
%Das Beispiel für \texttt{\textbackslash{}phantomsection} bitte im \LaTeX{}-Quellcode anschauen.

%Hier das Beispiel: Siehe Abschnitt \vref{hack1} und Abschnitt \vref{hack2}.

\section{Definitionen}
\begin{definition}[Title]
\label{def:def1}
Definition Text
\end{definition}

\Cref{def:def1} zeigt \ldots

\section{Fußnoten}
Fußnoten können mit dem Befehl \verb+\footnote{...}+ gesetzt werden\footnote{\label{fussnote}Diese Fußnote ist ein Beispiel.}. Mehrfache Verwendung von Fußnoten ist möglich indem man zu erst ein Label in der Fußnote setzt \verb+\footnote{\label{...}...}+ und anschließend mittels \verb+\cref{...}+ die Fußnote erneut verwendet\cref{fussnote}.

\section{Verschiedenes}
\label{sec:diff}
\ifdeutsch
Ziffern (123\,654\,789) werden schön gesetzt.
Entweder in einer Linie oder als Minuskel-Ziffern.
Letzteres erreicht man durch den Parameter \texttt{osf} bei dem Paket \texttt{libertine} bzw.\ \texttt{mathpazo} in \texttt{fonts.tex}.
\fi

\textsc{Kapitälchen} werden schön gesperrt...

\begin{compactenum}[I.]
\item Man kann auch die Nummerierung dank paralist kompakt halten
\item und auf eine andere Nummerierung umstellen
\end{compactenum}

\section{Weitere Illustrationen}
\Cref{fig:AnhangsChor,fig:AnhangsChor2} zeigen zwei Choreographien, die den Sachverhalt weiter erläutern sollen.
Die zweite Abbildung ist um 90 Grad gedreht, um das Paket \texttt{pdflscape} zu demonstrieren.

\begin{figure}
  \centering
  \includegraphics[width=\textwidth]{choreography.pdf}
  \caption{Beispiel-Choreographie I}
  \label{fig:AnhangsChor}
\end{figure}

\begin{landscape}
  \begin{figure}
    \centering
    \includegraphics[width=\textwidth]{choreography.pdf}
    \caption{Beispiel-Choreographie II}
    \label{fig:AnhangsChor2}
  \end{figure}
\end{landscape}


\iffalse

\clearpage

FIXME - This does not work with MiKTeX as of 2016-12-30

TODO- demonstrate rotating package

%hint by http://tex.stackexchange.com/a/3265/9075
%other option is to use changepage according to http://tex.stackexchange.com/a/2639/9075. This, however, has issues with landscape
\thispagestyle{empty}

\savegeometry{koma}

%If you only have height problems, this is not needed at all
\addtolength{\textwidth}{2cm}
\addtolength{\evensidemargin}{-1cm}

\begin{landscape}
  %sidewaysfigure
  \begin{figure}
    \centering
    \includegraphics[width=0.9\paperheight]{choreography.pdf}
    \caption{Beispiel-Choreographie, auf einer weißen Seite gezeigt wird und über die definierten Seitenränder herausragt}
  \end{figure}
\end{landscape}

%the original layout is restored.
%%\restoregeometry cannot be used as we use \addtolength
\loadgeometry{koma}

\fi


\IfFileExists{pgfplots.sty}{
\section{Plots with pgfplots}
Pgfplot ist ein Paket um Graphen zu plotten ohne den Umweg über gnuplot oder matplotlib zu gehen.
\begin{figure}[h]
\begin{center}
\begin{tikzpicture}
  \begin{axis}[xlabel=$x$,
               ylabel=$\sin(x)$]
    \addplot {sin(deg(x))};  % Sinus-Funktion zeichnen
  \end{axis}
\end{tikzpicture}
\end{center}
\caption{$\sin(x)$ mit pgfplots.}
\end{figure}
}{}

\IfFileExists{tikz.sty}{
\section{Figures with tikz}
TikZ ist ein Paket um Zeichnungen mittels Programmierung zu erstellen.
Dieses Paket eignet sich um Gitter zu erstellen oder andere regelmäßige Strukturen zu erstellen.
\begin{figure}[ht]
\begin{center}
\begin{tikzpicture}
  \draw(0,0) rectangle (4,4);
  \foreach \x in {0.5,1,1.5,2,2.5,3,3.5}
    \foreach \y in {0.5,1,1.5,2,2.5,3,3.5}
      \draw(\x,\y) circle (1pt);
\end{tikzpicture}
\end{center}
\caption{Eine tikz-Graphik.}\label{fig:tikz_example}
\end{figure}
}{}

\section{Schlusswort}
Verbesserungsvorschläge für diese Vorlage sind immer willkommen.
Bitte bei GitHub ein Ticket eintragen (\url{https://github.com/latextemplates/uni-stuttgart-computer-science-template/issues}).


\clearpage

%\printindex

\printbibliography

\ifdeutsch
Alle URLs wurden zuletzt am 17.\,03.\,2008 geprüft.
\else
All links were last followed on March 17, 2008.
\fi

\pagestyle{empty}
\renewcommand*{\chapterpagestyle}{empty}
\Versicherung
\end{document}
